%% LyX 2.4.3 created this file.  For more info, see https://www.lyx.org/.
%% Do not edit unless you really know what you are doing.
\documentclass[english]{article}
\usepackage[T1]{fontenc}
\usepackage{textcomp}
\usepackage[utf8]{inputenc}
\usepackage{url}
\usepackage{amsmath}
\usepackage{amsthm}
\usepackage{amssymb}
\usepackage{graphicx}
\usepackage{geometry}
\geometry{verbose,tmargin=3cm,bmargin=3cm,lmargin=3cm,rmargin=3cm}

\makeatletter

%%%%%%%%%%%%%%%%%%%%%%%%%%%%%% LyX specific LaTeX commands.
%% Because html converters don't know tabularnewline
\providecommand{\tabularnewline}{\\}

%%%%%%%%%%%%%%%%%%%%%%%%%%%%%% Textclass specific LaTeX commands.
\theoremstyle{plain}
\newtheorem{thm}{\protect\theoremname}
\theoremstyle{remark}
\newtheorem{rem}[thm]{\protect\remarkname}
\theoremstyle{definition}
\newtheorem{example}[thm]{\protect\examplename}

%%%%%%%%%%%%%%%%%%%%%%%%%%%%%% User specified LaTeX commands.
\usepackage[utf8]{inputenc}
\usepackage[T1]{fontenc}

\makeatother

\usepackage{babel}
\providecommand{\examplename}{Example}
\providecommand{\remarkname}{Remark}
\providecommand{\theoremname}{Theorem}

\begin{document}
\title{User Guide for the Discord Computer Olympiad Interface}
\author{Quentin Cohen-Solal\\
{\small quentin.cohen-solal@dauphine.psl.eu}}

\maketitle
This document explains how to use the three main features of the Discord
Computer Olympiad Interface (DCOI), namely:
\begin{itemize}
\item running a match in manual mode: a human manually enters the actions
of their program as text messages in the Discord conversation (Section~\ref{sec:Jeu-manuel}).
\item running a match in automatic mode: using a player Discord bot that
writes the game actions provided by your program in the Discord conversation
on your behalf (Section~\ref{sec:Jeu-automatis=0000E9-par}).
\item creating your own game server: to run a local beta test for the Computer
Olympiad or to organize your own competition (Section~\ref{sec:Cr=0000E9ation-d'un-serveur}).
\end{itemize}
%
If you wish to use only the automatic mode, it is nevertheless necessary
to read the following section. In addition to briefly describing how
to play in manual mode, it explains the DCOI in order to make the
following sections easier to understand --- including how to select
a game and start a match. This first section also covers the Free\_game
mode, which allows you to play (either in manual or automatic mode)
any perfect-information game, by delegating to the players the verification
of valid moves as well as the determination of the end of the game.
\begin{rem}
If you encounter any frustrations or bugs while reading this document
or using the DCOI, please don’t hesitate to let me know.
\end{rem}

This document is a translation of ``Guide d’utilisation du Discord
Computer Olympiad Interface'' (guide\_utilisateur\_DCOI.pdf). If
you have any problems with the translation, please consult this document
or the author of this document.

\tableofcontents{}

\section{Description of the DCOI and use in manual mode}\label{sec:Jeu-manuel}

To use the DCOI, you must be able to use Discord. Therefore, you need
to install the Discord application or use the web version, and create
an account if you haven’t already done so. (see \url{https://discord.com}).
This section begins by presenting the overall functioning of the system
(Section~\ref{subsec:Fonctionnement-global}), then explains how
to access the match channels (Section~\ref{subsec:L'application-Discord-et}),
followed by instructions on how to play a match (Section~\ref{subsec:Lancement-d'une-partie}),
including an explanation of the action syntax (Section~\ref{subsec:Syntaxe-des-actions}).
A few warnings are also described in Sections~\ref{subsec:Matchs-et-discussions}~,~\ref{subsec:Mise-en-garde}~and~\ref{subsec:Avertissement-sur-les}.
Finally, we briefly discuss the match logging system (Section~\ref{subsec:Logging-des-matchs}).

\subsection{Overall Functioning}\label{subsec:Fonctionnement-global}

\subsubsection{DCOI and Referee Bot}

The DCOI is both a Python library and a collection of ready-to-use
or extensible Python bots designed to create player bots and referee
bots, enabling gameplay and match management within a Discord conversation.

Players and program operators join the competition’s Discord server.
When match launch commands are entered in a Discord channel, the game
begins. It is then managed by a referee bot (set up by the organizers).
Players and player bots must post their game actions in the conversation
when prompted by the referee bot. The referee bot checks the legality
of moves, handles errors, tracks player time, and announces the end
of the game (either normal or due to a time limit being exceeded)
along with the winner.

\subsubsection{Focus on Player Bots}

All aspects related to Discord are automated through the Discord bots.
Users only need to connect their program to the player Discord bot,
which handles Discord events and posts messages in the conversation,
among other things. To interface the two programs, you can either
subclass a player bot and implement the plays method using DCOI’s
gameplay methods, or implement the Go Text Protocol (GTP) communication
protocol. In both cases, you may need to convert the DCOI game action
syntax into your program’s own syntax. Finally, you must create a
bot account and specify a configuration file to link the bot account
with your bot program.

\subsection{The Discord Application and Match Channels}\label{subsec:L'application-Discord-et}

\subsubsection{Discord: Servers and Channels}\label{subsec:Discord:-serveurs-et-salons}

We will now briefly review the different elements of the Discord application
that are needed to run a match.

A screenshot of the application is shown in Figure~\ref{fig:Discord-App}.
The first step is to join the competition server (servers on Discord
are called guilds). On your first access, click the “+” icon at the
bottom of the server list (see Figure~\ref{fig:Discord-servers-list}).
During the competition or the beta test, click “Join a Server”, then
enter the invitation link provided by the organizers. For the beta
test, you must join the beta test server: \url{https://discord.gg/mbHBdH3K}.
Once the server has been added to your account, you can return to
it at any time by clicking its icon, now visible in your server list
(see Figure~\ref{fig:Discord-servers-list}). After joining the server,
you need to enter a match channel by clicking on its name. The available
channels are displayed in the designated area (see Figure~\ref{fig:Discord-channel-list}).
You may also join the general or discussion channel to talk with other
participants, or any other channel --- for example, to ask for help.
As in any instant messaging platform, to write in the selected channel,
simply type in the text input area provided for this purpose (see
Figure~\ref{fig:Discord-input-area}). Once you press Enter, your
message will be sent and will appear in the area showing the conversation
history (the middle of the window --- see Figure~\ref{fig:Discord-conversation},
if needed).

\subsubsection{Matches and Discussions}\label{subsec:Matchs-et-discussions}

Note that you can also talk in the match channels, as long as your
message does not consist solely of a text-formatted action. Indeed,
if the match has started and you are one of the players, your message
will be interpreted as a game move.

For example, never write just “A1” or “A2-B3” during a conversation.
Instead, you may write something like “Why did you play A1?”, or even
just “A1.” or “A1?”.

In general, try to avoid chatting in the match channel as much as
possible, at least during ongoing matches, to prevent cluttering the
conversation.

\subsubsection{Warning About Editing or Deleting Messages}\label{subsec:Mise-en-garde}

Discord allows you to edit and delete your messages. During a match---whether
you or your bot are playing---never use this feature under any circumstances.
There is a small but non-negligible chance that doing so could cause
system instability, preventing the match from being completed.

If such an event occurs, you will be held responsible and will automatically
lose the match.

If you accidentally edit your move, immediately edit it again to restore
the original action --- this should resolve the issue without consequence.

If you accidentally delete your move, unfortunately, this action is
irreversible, and you must hope it does not affect the rest of the
game.

Note that the system’s default behavior is to ignore message deletions
and edits, so there is absolutely no benefit in attempting such actions.

In case of any problem, contact an organizer.

\begin{figure}
\begin{centering}
\includegraphics[scale=0.5]{interface}\caption{Discord App}\label{fig:Discord-App}
\par\end{centering}
\end{figure}

\begin{figure}

\begin{centering}
\includegraphics[scale=0.5]{interface_serveurs}\caption{Discord servers list (red box)}\label{fig:Discord-servers-list}
\par\end{centering}
\end{figure}
\begin{figure}
\centering{}\includegraphics[scale=0.5]{interface_salons}\caption{Discord channel list (red box)}\label{fig:Discord-channel-list}
\end{figure}
\begin{figure}
\centering{}\includegraphics[scale=0.5]{interface_zone_de_saisie}\caption{Discord input area for conversation and match playing (red box)}\label{fig:Discord-input-area}
\end{figure}
\begin{figure}
\centering{}\includegraphics[scale=0.5]{interface_conversation}\caption{Discord conversation / match history (red box)}\label{fig:Discord-conversation}
\end{figure}


\subsection{Starting a Match}\label{subsec:Lancement-d'une-partie}

We now move on to the core of the topic and explain what you need
to know --- and how to proceed --- to start a match.

\subsubsection{Choosing the Game and Other Parameters}

To choose the game for the match, type the following command in the
conversation:
\begin{center}
``!set game <GAME\_NAME>''.
\par\end{center}
\begin{example}
``!set game Clobber''.
\end{example}

\begin{rem}
If the command worked, the referee bot will respond in the conversation:
``Game set to <GAME\_NAME>''.
\end{rem}

To see the list of available games, you can type the following command
in the conversation:
\begin{center}
``!available\_games''.
\par\end{center}

If you need to adjust the total time allocated to each player per
match (default is 30 minutes per player per match), you can use the
command:
\begin{center}
``!time <T>'' 
\par\end{center}

where <T> is an integer indicating the time in seconds, or an integer
followed by one of the units “s”, “min”, or “h” to specify the unit
(for example, ``!time 20min'' ou ``!time 1h''). 
\begin{rem}
If you plan to play a series of matches on the same game, it is not
necessary to set the game before each launch.
\end{rem}


\subsubsection{Special Case: Free\_game Mode}

The DCOI provides the ability to play any perfect-information, deterministic
game using the “pseudo-game” called Free\_game. Note, however, that
--- while this is not a practical limitation --- players must take
turns alternately (i.e., never play twice in a row). This mode allows
matches to be played for such games but with the following trade-off:
no legality checking of moves and no automatic winner determination.
Players must therefore take turns entering the action “end” to signal
the end of the match. This feature should only be used if participants
are certain they are following the same rules and each have their
own game engine. To use Free\_game, type:
\begin{center}
``!set game Free\_game''.
\par\end{center}

The valid actions for this mode are all actions that follow the DCOI
action syntax (for example, “A1”, “B2-C3”, “D4-E17-H23”; see Section~\ref{subsec:Syntaxe-des-actions})
plus the action “end”. Yet, you can also add custom special actions
to Free\_game with the command:
\begin{center}
``!add\_freegame\_moves''. 
\par\end{center}

You can list all additional special actions of Free\_game with the
command:
\begin{center}
``!show\_extra\_freegame\_moves''. 
\par\end{center}

And you can remove all special actions from Free\_game with the command:
\begin{center}
``!clear\_freegame\_moves''. 
\par\end{center}

These commands must, of course, be used before the match begins.

\subsubsection{Starting the Match}

We now turn to the practical process of running a match. The first
step is to ask the referee bot to propose a game. To do this, use
the command:
\begin{center}
``!start @<Discord\_Player\_ID\_1> @<Discord\_Player\_ID\_2>''.
\par\end{center}

You can save time by using the alias “!s” instead of “!start”. To
enter this somewhat technical command, start typing “!start @” ---
Discord will then display a list of available user IDs. Use the up/down
arrow keys to select the ID of the first player in the match and press
Enter. If it is not in the list, you can start typing its name until
it appears. Note that simply typing the name does not work --- you
must always select it from the list and then confirm by pressing “Enter”.

Then type a space followed by “@” again, and select the second player
in the same way. If one of the players is using a bot, you must select
the bot’s ID, not the operator’s. Moreover, if one of the players
is a bot, that bot must already be running before the !start command
is sent.
\begin{rem}
There is a rare Discord bug in which some players or bots may temporarily
disappear from the “@” selector. This issue usually resolves itself,
but it can take hours or even days. However, it can easily be bypassed
in two ways:

• Simply use a different match channel.

• Or, write the command in another channel and copy-paste it into
the intended match channel.

Always keep at least one channel empty of messages, as it seems that
sending messages may trigger this strange bug.
\end{rem}

Once the command is executed, the referee displays the details of
the proposed match (see Figure~\ref{fig:waiting_start}).Participants
must then click the thumbs-up icon under the message to confirm the
match. If you are using a bot, this step is handled automatically
by the library --- you don’t need to do anything. In the current
configuration, players have 15 minutes to accept the match proposal.
When all players have clicked the thumbs-up, the match begins. If
you wish to decline the proposed match, click the thumbs-down icon
instead.
\begin{rem}
It is normal to see the number 1 next to the thumbs-up and thumbs-down
icons. These reactions were added by the referee bot to make them
visible and to make it easier for players to react.
\end{rem}

\begin{figure}
\begin{centering}
\includegraphics[scale=0.5]{start_attente_de_reponse}
\par\end{centering}
\caption{Example of the beginning of a match (example of waiting for player
confirmation to start the game)}\label{fig:waiting_start}
\end{figure}


\subsubsection{Progression and End of the Match}

During a match, the referee indicates when it is a player’s turn to
move. It writes a message in the conversation such as: ``@<PLAYER>
must play (he has <TIME> left)''. At that moment, the mentioned player
may enter in the conversation the action chosen by their program.
A player must not post any move before being explicitly told to play.
Figure~\ref{fig:D=0000E9but-d'une-partie} shows the beginning of
a match corresponding to the following sequence of actions: “J4-I5”
followed by “C5” (the first player plays twice in a row).

\begin{figure}
\begin{centering}
\includegraphics[scale=0.5]{start}
\par\end{centering}
\caption{Example of the beginning of a match}\label{fig:D=0000E9but-d'une-partie}
\end{figure}

If the action entered is incorrect, it is marked with a red square,
and the referee indicates the error and provides the list of legal
actions (see Figure~\ref{fig:Invalid-move}).
\begin{figure}
\centering{}\includegraphics[scale=0.5]{invalid_move}\caption{Example of invalid move}\label{fig:Invalid-move}
\end{figure}

When the match is over, the referee announces it and indicates the
winner (see Figure~\ref{fig:Example-of-endgame}).
\begin{figure}
\centering{}\includegraphics[scale=0.5]{endgame}\caption{Example of endgame}\label{fig:Example-of-endgame}
\end{figure}

Once the match is finished, you can start a new game using the ``!start''
command.

\subsubsection{Non-Standard Match Endings: Timeout / Forfeit}

We now turn to special match endings, triggered either by a player
exceeding their thinking time (default: 30 minutes per player per
game) or by a player’s decision.

If a player exceeds their allotted thinking time, the match ends and
that player loses the game (see Figure~\ref{fig:Example-of-timeout}).
\begin{figure}
\centering{}\includegraphics[scale=0.5]{timeout}\caption{Example of timeout}\label{fig:Example-of-timeout}
\end{figure}

If you need to stop an ongoing match, you can use the “!stop” command.
This will trigger a vote. Note that this command does not currently
work if you are using a Discord bot.

If you wish to forfeit a match, when it is your turn to play, you
can enter the action resign (type “resign”). This will immediately
end the match and award the victory to your opponent (without triggering
a vote).

If you are using a bot, use the “!resign” command instead. When it
is your bot’s turn, it will execute the resign action, which will
similarly end the match.

\subsubsection{Action Syntax (Text Format)}\label{subsec:Syntaxe-des-actions}

With the DCOI, it is important to distinguish between action syntax
and legal actions. In order for the DCOI to differentiate a normal
conversation from a game move, it requires a specific syntax. This
syntax includes moves like “A1”, “B2-C12”, and also longer sequences
such as “D5-E3-G15”, possibly combined with specific keywords, e.g.,
“queen-A8”. Note that “resign” is an example of a keyword common to
all games. To see the list of keywords for the current game, use the
command:
\begin{center}
!show\_move\_keywords.
\par\end{center}

Note that the general syntax for game actions is defined by the following
regular expression:
\begin{center}
({[}a-zA-Z{]}{[}0-9{]}\{1,2\} | keyword1 | ... | keyword\_n)(-({[}a-zA-Z{]}{[}0-9{]}\{1,2\}
| keyword1 | ... | keyword\_n)){*}.
\par\end{center}

In other words, a text-formatted action is a sequence of blocks of
length one or more, separated by hyphens, where each block is either
a coordinate (e.g., A12, B5, C18) or a keyword.

When a message is posted in the conversation, the referee bot checks
whether the text follows the expected syntax. If it does, and if it
is that player’s turn, the bot will attempt to play the action. If
the action is valid, it is executed; otherwise, the bot reports that
the action is illegal, as described previously. If the syntax is not
respected, the referee simply ignores the message.

\subsection{Match Logging}\label{subsec:Logging-des-matchs}

Matches are saved in .json files (in the directory discord\_interface/log/bot\_play\_log/),
using the following format:
\begin{center}
``play\_<PLAYER\_ID>\_Computer-Olympiad\_<CHANNEL\_NAME>\_<DATE>\_<TIME>.json''.
\par\end{center}

This file contains a human-readable Python dictionary with many attributes,
in particular:
\begin{itemize}
\item game\_name -- the name of the game,
\item moves -- the history of moves in the match,
\item players -- a list of PLAYER\_IDs,
\item winner -- True if you or your bot won the match.
\end{itemize}

\subsection{Warning About Matches That Are Too Short}\label{subsec:Avertissement-sur-les}

Note that there is a delay in action communication due to internet
latency (which depends on the quality of your connection), as well
as Discord’s latency, and to a lesser extent, the DCOI’s own processing
time. This delay is measured in seconds, typically around one second
on average. The DCOI is therefore not suitable for matches with an
average time of less than 1 second per action. Moreover, sending more
than one message per second by a bot or user violates Discord’s terms
of service, which interprets such behavior as flooding. This can lead
to various penalties, including bans, shadow bans, or slowed Discord
functions. However, in practice, sending actions at intervals of about
one second has never caused issues during our extensive testing.

\section{Automated Play via Discord Bot}\label{sec:Jeu-automatis=0000E9-par}

In this section, we explain the different ways to conduct a match
with the DCOI in automatic mode.

In all cases, automated play is carried out using a Discord player
bot. This requires creating a Discord bot account on the Discord Developer
Portal (see Section~\ref{subsec:Cr=0000E9ation-du-compte}). Next,
the native way to run matches in automatic mode is to extend one of
the DCOI Python Player classes (Section~\ref{subsec:Programmation-d'un-Bot-python}).
The other method, described in Section~\ref{subsec:Utilisation-du-Bot-GTP},
is to use a fully implemented bot that communicates with your program
via input/output (inter-process communication) using the Go Text Protocol
(GTP). This second approach has the advantage of not requiring any
programming with the DCOI library --- nor any knowledge of the Python
language. Finally, in Section~\ref{subsec:Fonctionnalit=0000E9s-avanc=0000E9es},
we introduce advanced features related to bots.

We begin by detailing the installation of the Python packages required
by the DCOI.

\subsection{Installation of Required DCOI Packages}

We will now detail the installation of DCOI. Note that Python 3.10+
is required to install and use DCOI.
\begin{rem}
If you are using Python for your projects, to avoid breaking dependencies,
it is recommended to use a Python virtual environment.
\end{rem}

To install the DCOI packages, you have two possible methods:

\paragraph{Quick method:}

Execute the following commands in your terminal:
\begin{itemize}
\item ``cd discord\_interface''
\item ``chmod +x install.sh''
\item ``bash install.sh''
\item (optional) Test the installation: run in the terminal: ``python3
test\_installation.py''.
\end{itemize}
\begin{rem}
If you move the discord\_interface folder to another location, you
will need to run ``bash install.sh'' again.
\end{rem}


\paragraph{Step-by-step method:}

The DCOI is based on the official Python library for Discord. To use
the DCOI library, you need to install it by running the following
command in your terminal: ``python3 -m pip install -U discord.py''.
You also need the numpy library to run the games: ``python3 -m pip
install -U numpy''. If you wish to use GTP, you will additionally
need the pexpect library to enable inter-process communication. Install
it with: ``pip install pexpect''. You can then test your installation
by running in the terminal: ``python3 test\_installation.py''.

Finally, you need to modify the Python path to include the parent
folder of “discord\_interface”:

\begin{verbatim}
UPPER_DIR_DCOI='<path_to_discord_interface>'
echo 'export PYTHONPATH=$PYTHONPATH:'$UPPER_DIR_DCOI >> ~/.bashrc
source ~/.bashrc
\end{verbatim}
\begin{rem}
If you move the discord\_interface folder, you will need to update
the PYTHON PATH with the new path.
\end{rem}


\subsection{Creating Your Player Discord Bot Account}\label{subsec:Cr=0000E9ation-du-compte}

We now turn to creating your bot account, which allows the bot program
to connect to the Discord application. More specifically, this series
of steps lets you retrieve various pieces of information. Some of
this information must be entered into the parameters.conf configuration
file located at the root of the discord\_interface folder. Creating
the account also provides a link that you must give to the organizer
so they add your bot to the competition’s Discord server.

\subsubsection{Creating the Bot Account and Inviting It to the Server}\label{subsec:Cr=0000E9er-le-compte-de-bot}
\begin{enumerate}
\item Go to the Discord Developer Portal website: \url{https://discord.com/developers/applications}.
Log in with your Discord account. Then click “New Application” at
the top right of the screen (see Figure~\ref{fig:New-Application}
if needed) and give your bot a name. You will then arrive at the bot’s
page.
\begin{enumerate}
\item If you encounter difficulties, try using a different web browser.
\end{enumerate}
\item Next, enable the “Message Content Intent” option in the Bot tab (see
Figure~\ref{fig:Bot-window-and}, if needed).
\item Then, retrieve the TOKEN of the bot --- this is essentially its password,
so it must be kept secret. Note it down, as we will need it to configure
the bot in the next section.
\begin{enumerate}
\item Go to the Bot tab (still on the bot page). In the TOKEN section, click
the “Reset Token” button (see Figure~\ref{fig:Reset-Token-button}
if needed).
\end{enumerate}
\item Go to the OAuth2 tab, then ``OAuth2 URL Generator''. Under the SCOPES
section, check the “bot” option.
\item 5. In the bot permissions subsection, check the following options:
\begin{enumerate}
\item \textquotedbl View Channels\textquotedbl ,
\item \textquotedbl Send Messages\textquotedbl , 
\item \textquotedbl Read Message History\textquotedbl , 
\item \textquotedbl Add Reactions\textquotedbl . 
\end{enumerate}
\item Finally, at the bottom of the page, in the ``Generated URL'' field,
copy the URL. Send this URL to the organizer so they can add your
bot to the competition’s Discord server.
\end{enumerate}
\begin{figure}
\centering{}\includegraphics[scale=0.15]{new_application}\caption{New application button (red box)}\label{fig:New-Application}
\end{figure}
\begin{figure}
\centering{}\includegraphics[scale=0.15]{\string"message content intent\string".png}\caption{Bot window and message content intent option to be activated (red
boxes)}\label{fig:Bot-window-and}
\end{figure}
\begin{figure}
\centering{}\includegraphics[scale=0.15]{token}\caption{Reset Token button (red box)}\label{fig:Reset-Token-button}
\end{figure}


\subsubsection{Retrieving Your Discord ID (Optional)}\label{subsec:R=0000E9cup=0000E9ration-de-votre}

Next, it is recommended to retrieve your Discord user ID to specify
it in the configuration file. This step is optional, but it allows
you to use commands that control your bot to access certain features.
For example, it is required to resume a match after a crash (see Section~\ref{subsec:Fonctionnalit=0000E9s-avanc=0000E9es}).
In this section, we explain how to obtain your Discord ID.

The first step is to enable Developer Mode on your Discord account.
To do this:
\begin{enumerate}
\item Open the Discord application.
\item Go to Settings (click the “gear” icon; see Figure~\ref{fig:Discord-parameters-button},
if needed). 
\item Click on the “Advanced” tab.
\item Enable the “Developer Mode” option.
\item Click on your username at the bottom right.
\item In the small pop-up window, click “Copy User ID” (see Figure~\ref{fig:Open-owner-window},
if needed).
\item Return to Settings and disable the Developer Mode option.
\end{enumerate}
\begin{figure}
\centering{}\includegraphics[scale=0.5]{interface_paramètres}\caption{Discord parameters button (red box)}\label{fig:Discord-parameters-button}
\end{figure}

\begin{figure}
\centering{}\includegraphics[scale=0.25]{advanced}\caption{Discord advanced window and developper mode option (red boxes)}\label{fig:Discord-advanced-window}
\end{figure}

\begin{figure}
\centering{}\includegraphics[scale=0.15]{\string"Owner id\string".png}\caption{Open owner window and copy Owner ID (red boxes)}\label{fig:Open-owner-window}
\end{figure}


\subsubsection{Specifying the Configuration File}

We now explain how to configure the parameters.conf file, located
at the root of the discord\_interface folder, which allows the bot
program to access Discord.
\begin{enumerate}
\item Set the value of PLAYER\_BOT\_DISCORD\_TOKEN to the TOKEN of your
bot account (retrieved in Section~\ref{subsec:Cr=0000E9er-le-compte-de-bot}).
\item Optional but recommended: set the value of OWNER\_ID to your Discord
user ID (retrieved in Section~\ref{subsec:R=0000E9cup=0000E9ration-de-votre}).
\end{enumerate}
To verify that the configuration file is correctly set, launch your
bot and check:

\begin{verbatim}
cd <discord_interface_path>/discord_interface 
python3 main_random_ai.py
\end{verbatim}

If the terminal shows the error “Bad OWNER\_ID”, you need to correct
the OWNER\_ID variable in ``parameters.conf''.

For a more advanced test, launch your bot and type in a channel on
the competition’s Discord server the following command:
\begin{center}
``!conf\_test''.
\par\end{center}
\begin{itemize}
\item If nothing happens, contact me.
\item If everything works correctly, “All is OK!” will appear in the terminal.
\item If an error occurs in the terminal (discord.errors.LoginFailure: Improper
token has been passed), the PLAYER\_BOT\_DISCORD\_TOKEN has been incorrectly
specified.
\end{itemize}

\subsection{Programming a Discord Bot Using Python}\label{subsec:Programmation-d'un-Bot-python}

We now explain how to program a Discord bot in Python. You need to
inherit one of the Python Player classes provided by the DCOI library.
Here, we describe the simplest way to do this.

\paragraph{Bot programming}

To program your bot, you need to inherit the BasicPlayer class from
the file: ``discord\_interface/player/model/basic\_player.py''.
The only thing you need to do is define the method: my\_plays(self,
game\_history, time\_left=inf, opponent\_time\_left=inf). This method
should return the action that your program wants to play in the current
game state. The current state can be reconstructed from the game\_history
parameter, which contains the list of moves in text format played
since the start of the match. You can also use the native game methods
provided by the DCOI, which allows you to avoid implementing the game
logic yourself. In this second case, you will need the methods described
in Table~\ref{tab:Elementary-game-methods}.

\begin{table}
\begin{centering}
\begin{tabular}{|c|c|}
\hline 
game methods of any player & \tabularnewline
\hline 
\hline 
self.game.get\_numpy\_board() & return a numpy array representing the game board\tabularnewline
\hline 
self.game.get\_current\_player() & returns $i$ if the current player is the player $i$ ($i\in\mathbb{N}$) \tabularnewline
\hline 
self.game.textual\_legal\_moves() & list of valid actions in text format (e.g. ``A1'' or ``A2-B3'')\tabularnewline
\hline 
self.game.ended() & return True if the game is ended\tabularnewline
\hline 
self.game.winner & return the index $i$ of the winner player ($i\in\mathbb{N}$) \tabularnewline
\hline 
self.game.textual\_plays(move) & apply the move in the current game state (i.e.self.game)\tabularnewline
\hline 
self.game.undo() & undo the last action of the current game state (i.e.self.game)\tabularnewline
\hline 
\end{tabular}
\par\end{centering}
\caption{Elementary methods of any Player Python object for programming an
AI player.}\label{tab:Elementary-game-methods}

\end{table}

Pay attention to the time\_left parameter: it indicates the remaining
time your program has for the rest of the match. If this value reaches
0, you will lose the game.

An example of inheritance is provided in the TextualRandomAI class
in the file:
\begin{center}
``discord\_interface/player/instances/textual\_random\_ai.py''. 
\par\end{center}

This bot plays randomly, except when there is a winning move among
the available actions. In that case, it plays the winning move.

\paragraph{Launching the bot}

Once your bot account is created, the configuration file is set up,
and your Player class is inherited, all that remains is to launch
your bot in a main script. To do this, use the method: bot\_starting(AI\_Class)
from the file: discord\_interface/player/model/bot\_launcher.py with
AI\_Class set to your AI Python Player class. An example of a bot
launcher program is provided in the file:
\begin{center}
``discord\_interface/main\_random\_ai.py''.
\par\end{center}

All that remains for you to do is: ``python3 <your\_main>.py'' in
the folder ``discord\_interface''. For example:

\begin{verbatim}
cd <discord_interface_path>/discord_interface 
python3 main_textual_random_ai.py
\end{verbatim}

The bot now waits for the command: !s @<Bot\_name\_1> @<Bot\_name\_2>
to be executed and for the referee bot to display the match information.
At that point, the bot performs its initialization and adds its “thumbs
up.” The match then begins, and the bot plays automatically whenever
it is its turn, until the end of the game.
\begin{rem}
If you used a Python virtual environment during installation, remember
to activate it before launching the bot.
\end{rem}


\paragraph{Advanced Features}

Finally, optionally, if you need advanced features, such as code to
execute at the initialization or end of the match, you can inherit
from the class AdvancedBasicPlayer (file: discord\_interface/player/model/advanced\_basic\_player.py)
and define the methods: my\_initialisation(game\_name) and my\_end().

\subsection{Using the Generic Player Discord Bot via the GTP}\label{subsec:Utilisation-du-Bot-GTP}

We now explain how to run automatic matches using GTP. Information
about GTP is available, for example, on this website:: \url{https://www.lysator.liu.se/~gunnar/gtp/gtp2-spec-draft2/gtp2-spec.html}.
In Section~\ref{subsec:Sp=0000E9cification-des-commandes}, we will
show how to specify a GTP program. Here, we begin by describing how
to launch the GTP bot, assuming you already have a program specified
in the GTP format. Finally, we will cover an advanced feature that
allows you to display information from your program in the terminal,
(Section~\ref{subsec:Afficher-des-informations}) as well as how
to handle errors in your program when using GTP (Section~\ref{subsec:Gestion-des-erreurs}).

\subsubsection{Starting the GTP Bot}

We begin by explaining the prerequisites for launching your GTP bot.
First, you need to connect your program to the GTP bot. To do this,
you must specify three parameters:
\begin{itemize}
\item ``program\_name'' : the name of your player program,
\item ``program\_arguments'': the list of arguments for your program (separated
by spaces),
\item ``program\_directory'': the absolute path to your program (or the
path relative to the discord\_interface folder; for example, you can
use an empty string if the program is located at the root of the discord\_interface
folder).
\end{itemize}
These variables must be specified in the parameters.conf file at the
root of the discord\_interface folder.

To run your program, all you need to do is:
\begin{center}
``python3 main\_gtp\_ai\_from\_conf.py''
\par\end{center}

in the terminal from the root of the discord\_interface folder.
\begin{rem}
If you are comfortable with Python, you can also modify and run the
main\_gtp\_ai.py file to avoid editing the configuration file.
\end{rem}


\subsubsection{Specifying the GTP Commands}\label{subsec:Sp=0000E9cification-des-commandes}

In order to use the GTP-based bot, your program must read GTP commands
from standard input (which will be sent by the GTP bot) and write
the results of these commands to standard output (which will then
be read and processed by the GTP bot).

The required GTP commands to run the GTP bot, which need to be implemented
in your program, are described in Table~\ref{tab:List-of-GTP-commands}.
To verify that your GTP command implementation is correct, you can
run the following test program in the terminal: ``python3 test\_gtp\_ai\_from\_conf.py''.

\begin{table}
\begin{centering}
\begin{tabular}{|c|c|c|}
\hline 
GTP command & return & effect\tabularnewline
\hline 
\hline 
clear\_board & $=$ & restart the game\tabularnewline
\hline 
undo & $=$ & cancel the last move\tabularnewline
\hline 
play <player> <move> & $=$ & plays <move> for <player>\tabularnewline
\hline 
time\_left <player> <time> & $=$ & inform the time left in seconds\tabularnewline
\hline 
genmove <player> & $=$<move> & return the <move> to play and play it\tabularnewline
\hline 
quit & $=$ & shutdown the program\tabularnewline
\hline 
\hline 
player & $=$<player> & return the current player\tabularnewline
\hline 
game <game> & $=$ & initialization for <game>\tabularnewline
\hline 
\end{tabular}
\par\end{centering}
\caption{List of GTP commands to implement, the syntax of their return and
their expected effect (<move> is an action in text format (see the
syntax in Section~\ref{subsec:Syntaxe-des-actions}) ; <player> is
an integer $n\in\mathbb{N}$ ; <game> is the game name of the match).
Remark: commands ``game'' and ``player'' are not part of standard
GTP; they are optional (make them only return ``=''); ``game''
is required only for multi-games participation ; ``player'' is required
only for profiling (see Section~\ref{subsec:Bot-profiling})..}\label{tab:List-of-GTP-commands}

\end{table}

If needed, an example of a Python implementation of GTP commands is
available in the file: ``discord\_interface/player/instances/autres/gtp\_random\_ai.py''.
\begin{rem}
Example of GTP communication between the bot and the player program:

\begin{verbatim}
genmove 0
=A2-A3
play 1 A7-A6
=
genmove 0
=A3-A4
...
\end{verbatim}
\end{rem}

\begin{rem}
If you want your bot to be able to resign automatically, it simply
needs to return ``=resign'' in response to the genmove command at
the appropriate time.
\end{rem}

\begin{rem}
If you want to resign your bot manually, use the already functional
command !resign.
\end{rem}


\subsubsection{Displaying Information in the Terminal}\label{subsec:Afficher-des-informations}

Finally, we briefly discuss an option that allows you to display information
from your program in the terminal when using the GTP bot.

Setting up this feature is very simple. For each GTP command return,
simply concatenate the return with the \# symbol followed by the line
you want to display in the terminal.
\begin{example}
If you want to display in the terminal: ``The move was correctly
cancelled'' when the undo command is executed, the return of undo
should be: ``=\#The move was correctly cancelled''.
\end{example}

Note that the line to display in the terminal must not contain line
breaks. However, you can create line breaks by adding multiple \#
symbols. Each additional \# will be interpreted as a line break.
\begin{example}
If the return of the genmove command is: ``=<move>\#Calculating the
next move completed\#Move played: <move>\#Calculation time 15 seconds'',
this will display the following in the terminal:
\end{example}

\begin{verbatim}
Calculating the next move completed
Move played: <move>
Calculation time 15 seconds
\end{verbatim}

\subsubsection{Error Handling with GTP}\label{subsec:Gestion-des-erreurs}

If you want error handling with GTP, meaning that your program notifies
the Discord bot when an error occurs, you must use the standard GTP
syntax designed for this purpose. Specifically, when an error occurs
and needs to be communicated, write to standard output but modify
the output format by replacing ``=<output\_line>'' with ``?<output\_line>''.
The presence of the ``?'' allows the Discord bot to trigger an error,
which will be displayed in the terminal. Additional information will
also appear in the terminal, including:
\begin{itemize}
\item The name of the command that caused the ``?'',
\item The output of your program,
\item Any comments displayed using the feature described in the previous
section.
\end{itemize}

\subsection{Advanced Features}\label{subsec:Fonctionnalit=0000E9s-avanc=0000E9es}

We now describe several advanced features of player bots.

\subsubsection{Crash Management}

If for any reason your bot crashes and needs to be restarted, to resume
the ongoing match, after relaunching your bot in the terminal, you
must use in the Discord conversation the command:
\begin{center}
``!continue''.
\par\end{center}

Its alias is ``!c''.

Before the competition, I suggest forcing a bot kill during a match,
then relaunching it and verifying that the !continue command works
correctly. This ensures you can resume a match in case of a problem.

If the referee bot does not respond, contact a competition official
to restart the referee and also execute the !continue command.

\subsubsection{Bot Profiling}\label{subsec:Bot-profiling}

Profiling can be performed using the command: ``!p''. 
\begin{rem}
If you are using a GTP bot, some of the profiling values provided
may be incorrect if the GTP ``player'' command has not been implemented
or has been implemented incorrectly.
\end{rem}


\section{Creating a Personal Discord Game Server}\label{sec:Cr=0000E9ation-d'un-serveur}

In this section, we explain how to create your own personal game server,
either for a local beta test of the Computer Olympiad or to organize
your own competition. Several steps must be completed:
\begin{enumerate}
\item Creating a Discord server (Section~\ref{subsec:Cr=0000E9ation-d'un-serveur}),
\item Creating the “bot account” for the referee bot (Section~\ref{subsec:Cr=0000E9ation-du-compte-arbitre}),
\item Inviting players to the server (Section~\ref{subsec:Invitation-des-joueurs}),
\item Inviting bots to the server (Section~\ref{subsec:Invitation-des-bots}),
\item Launching the referee bot (Section~\ref{subsec:Lancement-du-bot}).
\end{enumerate}
Note that there can be only one referee bot per channel, and you must
specify the corresponding channel for each referee bot. This will
be addressed in the following section. Therefore, you need multiple
referee bots if you want to run multiple matches in parallel.

\subsection{Creating a Discord Server}\label{subsec:Cr=0000E9ation-d'un-serveur}

We begin by detailing how to create a Discord server:
\begin{enumerate}
\item In the vertical server bar on the left of the Discord application:
\begin{enumerate}
\item Click the “+”, then “Create My Own” (choose your options and server
name).
\end{enumerate}
\item Enable ``Developer Mode`` (steps 1 to 4 in Section~\ref{subsec:R=0000E9cup=0000E9ration-de-votre}).
\item Left-click the icon of the server you just created in the server bar.
\item Click “Copy Server ID”.
\item For a Computer Olympiad beta test:
\begin{enumerate}
\item In the file parameters.conf:
\begin{enumerate}
\item copy the line starting with:
\begin{enumerate}
\item ``BETA\_TEST\_COMPUTER\_OLYMPIAD\_GUILD\_ID=''.
\end{enumerate}
\item Comment out the original line with \#.
\item Replace the value of BETA\_TEST\_COMPUTER\_OLYMPIAD\_GUILD\_ID with
the ID of the created server.
\item Ensure that in parameters.conf, BETA\_TEST\_MODE=True.
\end{enumerate}
\item For each referee bot (index \textit{i}):
\begin{enumerate}
\item In the server, create a channel named ``match\_\textit{i}''.
\item Right-click the channel and select Copy ID.
\item In parameters.conf
\begin{enumerate}
\item paste it as the value of REFEREE\_BOT\_\textit{i}\_DISCORD\_CHANNEL.
\end{enumerate}
\end{enumerate}
\end{enumerate}
\item For a server used for other purposes:
\begin{enumerate}
\item In parameters.conf:
\begin{enumerate}
\item Duplicate the line starting with:
\begin{enumerate}
\item ``COMPUTER\_OLYMPIAD\_GUILD\_ID=''.
\end{enumerate}
\item Comment out the original line with \#.
\item Replace the value of ``COMPUTER\_OLYMPIAD\_GUILD\_ID'' with the
ID of the created server.
\item Ensure that in parameters.conf, BETA\_TEST\_MODE=False.
\end{enumerate}
\item For each referee bot (index \textit{i}):
\begin{enumerate}
\item In the server, create a channel named ``match\_\textit{i}''.
\item Right-click the channel and select Copy ID.
\item In parameters.conf
\begin{enumerate}
\item paste it as the value of BETA\_TEST\_REFEREE\_BOT\_\textsl{i}\_DISCORD\_CHANNEL
\end{enumerate}
\end{enumerate}
\end{enumerate}
\item Deactivate ``Developer Mode''
\end{enumerate}
\begin{rem}
If you later participate in the Computer Olympiad beta test, remember
to remove the added line and uncomment the original.
\end{rem}

\begin{rem}
If you later participate in the Computer Olympiad, remember to set
BETA\_TEST\_MODE=False.
\end{rem}


\subsection{Creating the Referee “Bot Account”}\label{subsec:Cr=0000E9ation-du-compte-arbitre}

We now explain how to create a “bot account” for a referee bot. Note
that, you need to create a bot account for each referee bot.

The procedure is identical to the one in Section~\ref{subsec:Cr=0000E9er-le-compte-de-bot}
except for step 5, where you must additionally enable the “Manage
Messages” permission and that for each referee bot, the corresponding
token must be set for REFEREE\_BOT\_\textit{i}\_DISCORD\_TOKEN in
parameter.conf.

\subsection{Inviting Players to the Server}\label{subsec:Invitation-des-joueurs}

We now explain how to invite human players and bot operators to your
server:
\begin{enumerate}
\item Right-click the icon of your server in the vertical server bar on
the left.
\item Click “Invite People”.
\item Click the “Copy” button.
\item Send the copied link to the people you want to invite.
\end{enumerate}
The invited participants must then follow the protocol described in
Section~\ref{subsec:Discord:-serveurs-et-salons} using this link
instead of the Computer Olympiad server link.

\subsection{Inviting Bots to the Server}\label{subsec:Invitation-des-bots}

We now explain how to invite player bots and referee bots to your
server. We assume that you have followed Section~\ref{subsec:Cr=0000E9ation-d'un-serveur}
and obtained the referee bot invitation URL, and that participants
have also provided their invitation URLs after following Section~\ref{subsec:Cr=0000E9ation-du-compte}.
For each URL:
\begin{enumerate}
\item Paste the URL into your browser.
\item Log in to your Discord account (if required).
\item Choose your server from the list of available servers.
\item Click the “Continue” button.
\item If it is someone else’s bot:
\begin{enumerate}
\item Verify that the requested permissions match those specified in the
corresponding bot account creation section (player: Section~\ref{subsec:Cr=0000E9ation-du-compte}
/ referee: Section~\ref{subsec:Cr=0000E9ation-d'un-serveur}).
\end{enumerate}
\item Click the “Authorize” button.
\end{enumerate}

\subsection{Starting the Referee Bot}\label{subsec:Lancement-du-bot}

We explain in this section how to launch the referee bot. To do so,
run the following command in the terminal from the root of the discord\_interface
folder:
\begin{center}
``python3 main\_referee\_1.py''
\par\end{center}

Manual-mode players should then follow Section~\ref{sec:Jeu-manuel}
and automatic-mode players should follow Section~\ref{sec:Jeu-automatis=0000E9-par}.
\begin{rem}
To start a second referee bot, run: ``python3 main\_referee\_2.py.
\end{rem}

\begin{rem}
If the referee bot program crashes during a match, restart it and
then execute the !continue command in the match channel.
\end{rem}

\begin{rem}
If one of the player bots crashes during a match and restarting it
and using !continue does not fix the issue, you can try killing the
referee bot, restarting it, and using the !continue command (if you
suspect the problem originated from or propagated to the referee bot).
\end{rem}


\end{document}
