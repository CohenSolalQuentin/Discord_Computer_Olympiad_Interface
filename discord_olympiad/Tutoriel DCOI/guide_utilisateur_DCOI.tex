%% LyX 2.4.3 created this file.  For more info, see https://www.lyx.org/.
%% Do not edit unless you really know what you are doing.
\documentclass[french]{article}
\usepackage[T1]{fontenc}
\usepackage[utf8]{inputenc}
\usepackage{url}
\usepackage{amsmath}
\usepackage{amsthm}
\usepackage{amssymb}
\usepackage{graphicx}
\usepackage{geometry}
\geometry{verbose,tmargin=3cm,bmargin=3cm,lmargin=3cm,rmargin=3cm}

\makeatletter

%%%%%%%%%%%%%%%%%%%%%%%%%%%%%% LyX specific LaTeX commands.
%% Because html converters don't know tabularnewline
\providecommand{\tabularnewline}{\\}

%%%%%%%%%%%%%%%%%%%%%%%%%%%%%% Textclass specific LaTeX commands.
\theoremstyle{plain}
\newtheorem{thm}{\protect\theoremname}
\theoremstyle{remark}
\newtheorem{rem}[thm]{\protect\remarkname}
\theoremstyle{definition}
\newtheorem{example}[thm]{\protect\examplename}

%%%%%%%%%%%%%%%%%%%%%%%%%%%%%% User specified LaTeX commands.
\usepackage[utf8]{inputenc}
\usepackage[T1]{fontenc}

\makeatother

\usepackage{babel}
\providecommand{\examplename}{Exemple}
\providecommand{\remarkname}{Remarque}
\providecommand{\theoremname}{Théorème}

\begin{document}
\title{Guide d'utilisation du Discord Computer Olympiad Interface}
\author{Quentin Cohen-Solal\\
{\small quentin.cohen-solal@dauphine.psl.eu}}

\maketitle
Ce document explique comment utiliser les trois fonctionnalités principales
du Discord Computer Olympiad Interface (DCOI), à savoir:
\begin{itemize}
\item effectuer un match en mode manuel : un humain entre textuellement
les actions de son programme dans la conversation discord (Section~\ref{sec:Jeu-manuel}).
\item effectuer un match en mode automatique: utiliser un bot Discord joueur
qui écrit à votre place les actions de jeu indiqué par votre programme
dans la conversation discord (Section~\ref{sec:Jeu-automatis=0000E9-par}).
\item créer son serveur de jeu personnel, pour faire une beta test local
pour les Computer Olympiad ou pour organiser sa propre compétition
(Section~\ref{sec:Cr=0000E9ation-d'un-serveur}).
\end{itemize}
%
Si l'on souhaite utiliser uniquement le mode automatique, il est toutefois
nécessaire de lire la section suivante, qui en plus de décrire succinctement
comment jouer en mode manuel, explique le DCOI afin de faciliter la
compréhension des sections suivantes, ce qui inclut comment choisir
un jeu et lancer une partie. Cette première section traite également
du mode de jeu Free\_game qui permet de jouer (en mode manuel ou automatique)
à n'importe quel jeu à information parfaite, en déléguant aux joueurs
la détermination et la vérification des actions valides ainsi que
de la fin de la partie.
\begin{rem}
si vous rencontrez des frustrations ou des bugs à la lecture de ce
document ou à l'utilisation du DCOI, n'hésitez pas à me les signaler.
\end{rem}

\tableofcontents{}

\section{Description du DCOI et utilisation en mode manuel}\label{sec:Jeu-manuel}

Pour utiliser le DCOI, vous devez pouvoir utiliser Discord. Vous devez
donc installer l'application ou utiliser la version web de Discord
 et créer un compte, si ce n'est pas déjà fait (voir \url{https://discord.com}).
Cette section commence par présenter le fonctionnement global (Section~\ref{subsec:Fonctionnement-global}),
puis présente l'accès aux salons des matchs (Section~\ref{subsec:L'application-Discord-et}),
suivi du mode d'emploi pour réaliser une partie (Section~\ref{subsec:Lancement-d'une-partie}),
en expliquant notamment la syntaxe des actions (Section~\ref{subsec:Syntaxe-des-actions}).
Quelques mises en garde sont également décrites dans les Sections~\ref{subsec:Matchs-et-discussions}~,~\ref{subsec:Mise-en-garde}~et~\ref{subsec:Avertissement-sur-les}.
Enfin, nous parlons brièvement du système de logging des matchs (Section~\ref{subsec:Logging-des-matchs}).

\subsection{Fonctionnement global}\label{subsec:Fonctionnement-global}

\subsubsection{DCOI et bot arbitre}

Le DCOI est à la fois une librairie Python et un ensemble de bots
Python tout prêt ou à hériter pour réaliser des bots joueurs et des
bots arbitres, permettant de jouer et de gérer des matchs au sein
d'une discussion Discord.

Les joueurs et opérateurs rejoignent le serveur Discord de la compétition.
Lorsque les commandes de lancement de matchs sont inscrites dans la
conversation d'un salon Discord, la partie commence. Celle-ci est
ainsi gérée par un bot arbitre (mis en place par les organisateurs).
Les joueurs et bots joueurs doivent alors inscrire dans la conversation
leurs actions de jeu lorsque le bot arbitre les y invitent. Le bot
arbitre vérifie la légalité des coups, gère les erreurs, chronomètre
le temps des joueurs et indique une fin de partie (normal ou du à
un dépassement du temps autorisé) et le vainqueur de la partie. 

\subsubsection{Focus sur les bots joueurs}

Toute la partie concernant Discord est automatisée avec les bots Discord.
Les utilisateurs n'ont qu'une chose à faire : connecter leur programme
au bot Discord joueur, qui s'occupera de gérer les événements Discord
et d'écrire dans la conversation par exemple. Pour faire l'interface
entre les deux programmes, il faut soit hériter un bot joueur et implémenter
la méthode ``plays'' en utilisant les méthodes de jeux du DCOI soit
implémenter le protocole de communication Go Text Protocol. Dans les
deux cas, il peut aussi y avoir besoin d'effectuer une conversion
de la syntaxe des actions de jeu du DCOI dans la syntaxe de votre
programme. Enfin, il faut créer un compte de bot et spécifier un fichier
de configuration pour faire le lien entre le compte de bot et le programme
de bot.

\subsection{L'application Discord et les salons des matchs}\label{subsec:L'application-Discord-et}

\subsubsection{Discord: serveurs et salons}\label{subsec:Discord:-serveurs-et-salons}

Nous passons rapidement en revue les différents éléments de l'application
Discord dont nous avons besoin pour réaliser un match.

Une capture d'écran de l'application est disponible dans la Figure~\ref{fig:Discord-App}.
La première étape consiste à rejoindre le serveur de la compétition
(les serveurs sur discord s'appellent des guildes). Lors du premier
accès, il faut cliquer sur le ``+'' en bas de la liste des serveurs
(voir Figure~\ref{fig:Discord-servers-list}). Lors de la compétition
ou de la beta test, cliquez sur \textquotedbl rejoindre un serveur\textquotedbl{}
puis entrer le lien du serveur fournit par les organisateurs. Pour
la beta test, il faut rejoindre le serveur \url{https://discord.gg/mbHBdH3K}.
Une fois le serveur ajouté à votre compte, vous pouvez à tout moment
y retourner en cliquant sur l'icone du serveur désormais présente
dans la liste des serveurs (voir Figure~\ref{fig:Discord-servers-list}).
Une fois rejoint le serveur, il faut rejoindre le salon du match,
en cliquant sur son nom. Les salons disponibles se trouvent dans la
zone prévue à cet effet (voir la Figure~\ref{fig:Discord-channel-list}).
Vous pouvez aussi rejoindre le salon général / discussion pour parler
avec les autres participants ou tout autre salon, notamment pour demander
de l'aide. Comme dans n'importe quelle messagerie instantanée, pour
écrire dans le salon choisi, il suffit d'écrire en utilisant votre
clavier dans la zone prévue à cette effet (Figure~\ref{fig:Discord-input-area}).
Une fois appuyé sur entrée, votre message est envoyé et rejoindra
la zone affichant l'historique de la conversation (milieu de la fenêtre,
si besoin voir la Figure~\ref{fig:Discord-conversation}).

\subsubsection{Matchs et discussions}\label{subsec:Matchs-et-discussions}

Notez que vous pouvez aussi parler dans les salons des matchs, à condition
de ne pas écrire comme seul élément de votre phrase une action au
format textuel. En effet, si le match a commencé et que vous faites
parti des joueurs, votre phrase sera interprétée comme une action
de jeu. Par exemple, n'écrivez jamais juste ``A1'' ou ``A2-B3''
lors d'une conversation. Mais vous pouvez écrire à la place ``Pourquoi
tu as joué A1?'' ou même juste ``A1.'' ou ``A1?''. De manière
générale, évitez autant que possible de discuter dans le salon des
matchs pour ne pas le polluer, au moins durant les matchs. 

\subsubsection{Mise en garde sur la modification / suppression des messages}\label{subsec:Mise-en-garde}

Discord vous permet de modifier et supprimer vos messages. Lors d'un
match que cela soit vous ou votre bot, n'utilisez sous aucun prétexte
cette fonctionnalité. Il y a une chance faible mais non négligeable
que cela entraine une instabilité du système empêchant de finir la
partie. Si un tel événement se produit, vous serez considéré responsable
et vous perdrez la partie. Si vous modifiez votre action par erreur,
remodifiez votre message pour remettre l'action initiale. Il ne devrait
alors y avoir aucun problème. Si vous supprimez votre action par erreur,
c'est malheureusement définitif et il faut espérer que cela n'impactera
pas la suite de la partie. En cas de soucis, contactez un organisateur. 

\begin{figure}
\begin{centering}
\includegraphics[scale=0.5]{interface}\caption{Discord App}\label{fig:Discord-App}
\par\end{centering}
\end{figure}

\begin{figure}

\begin{centering}
\includegraphics[scale=0.5]{interface_serveurs}\caption{Discord servers list (red box)}\label{fig:Discord-servers-list}
\par\end{centering}
\end{figure}
\begin{figure}
\centering{}\includegraphics[scale=0.5]{interface_salons}\caption{Discord channel list (red box)}\label{fig:Discord-channel-list}
\end{figure}
\begin{figure}
\centering{}\includegraphics[scale=0.5]{interface_zone_de_saisie}\caption{Discord input area for conversation and match playing (red box)}\label{fig:Discord-input-area}
\end{figure}
\begin{figure}
\centering{}\includegraphics[scale=0.5]{interface_conversation}\caption{Discord conversation / match history (red box)}\label{fig:Discord-conversation}
\end{figure}


\subsection{Lancement d'une partie}\label{subsec:Lancement-d'une-partie}

Nous entrons maintenant dans le vif du sujet et présentons ce qu'il
faut savoir et comment faire pour lancer une partie.

\subsubsection{Choix du jeu et des autres paramètres}

Pour choisir le jeu du match, écrivez dans la discussion : 
\begin{center}
``!set game <GAME\_NAME>''.
\par\end{center}
\begin{example}
``!set game Clobber''.
\end{example}

\begin{rem}
Si cela a marché, le bot arbitre écrira dans la conversation: ``Game
set to <GAME\_NAME>''.
\end{rem}

Pour connaitre les jeux disponibles, vous pouvez écrire dans la conversation
la commande suivante : 
\begin{center}
``!available\_games''.
\par\end{center}

Si vous avez besoin de modifier le temps total accordé à chaque joueur
par partie (défaut 30 minutes par joueur par partie), vous pouvez
utiliser la commande: 
\begin{center}
``!time <T>'' 
\par\end{center}

avec <T> un nombre entier indiquant le temps en secondes ou un entier
collé à un des mots suivants ``s'', ``min'', ``h'' pour spécifier
l'unité (par exemple ``!time 20min'' ou ``!time 1h''). 
\begin{rem}
Si vous avez besoin de jouer une série de matchs sur le même jeu,
il n'est pas nécessaire de fixer le jeu avant chaque lancement.
\end{rem}


\subsubsection{Cas particulier du mode de Jeu Free\_game}

Le DCOI offre la possibilité de jouer à n'importe quel jeu à information
parfaite sans hasard grâce au ``pseudo-jeu'' nommé ``Free\_game''.
Notez toutefois que, sans que cela soit restrictif en pratique, les
joueurs doivent jouer alternativement (c'est-à-dire jamais deux fois
de suite). Ce dernier permet ainsi de jouer des matchs sur de tels
jeux mais avec la contre-partie suivante : il n'y a pas de vérification
des coups licites ni de détermination de vainqueur et les participants
doivent jouer (écrire) à tour de rôle l'action ``end'' pour mettre
fin à la partie. Cette fonctionnalité n'est donc à utiliser que si
les participants sont sûr de jouer selon les mêmes règles et qu'ils
disposent évidemment chacun leur propre moteur de jeu. Pour utiliser
Free\_game, faites:
\begin{center}
``!set game Free\_game''.
\par\end{center}

Les actions autorisées avec ce jeu sont toutes les actions vérifiant
la syntaxe des actions du DCOI (par exemple ``A1'', ``B2-C3'',
``D4-E17-H23'' ; voir la Section \ref{subsec:Syntaxe-des-actions})
auquel l'action ``end'' est rajoutée. Toutefois, vous pouvez aussi
rajouter des actions spéciales au jeu Free\_game avec la commande: 
\begin{center}
``!add\_freegame\_moves''. 
\par\end{center}

Vous pouvez afficher les actions spéciales additionnelles de Free\_game
avec la commande:
\begin{center}
``!show\_extra\_freegame\_moves''. 
\par\end{center}

Et vous pouvez retirer toutes les actions spéciales de Free\_game
avec la commande: 
\begin{center}
``!clear\_freegame\_moves''. 
\par\end{center}

Ces commandes sont bien évidemment à utiliser avant le début de la
partie.

\subsubsection{Déclanchement de la partie}

Nous nous intéressons maintenant sur comment réaliser en pratique
le match. La première étape est de demander à l'arbitre de proposer
une partie. Il faut utiliser la commande: 
\begin{center}
``!start @<Discord\_Player\_ID\_1> @<Discord\_Player\_ID\_2>''.
\par\end{center}

Pour gagner du temps, vous pouvez utiliser l'alias ``!s'' à la place
de ``!start''. Pour écrire cette commande un peu technique, commencez
par écrire ``!start @'', Discord va alors vous afficher une liste
des identifiants disponibles. Utiliser flèches du haut/bas pour choisir
l'identifiant du premier joueur du match et faites ``entrée''. Ecrivez
ensuite `` @'' et sélectionnez de la même façon le second joueur.
Si l'un des joueurs utilise un bot, c'est l'identifiant de son bot
qu'il faut utiliser et non l'identifiant de l'opérateur du bot. En
outre, si l'un des joueurs utilise un bot, ce dernier doit être lancé
avant que la commande !start ne soit écrite. 
\begin{rem}
Notez qu'il y a un bug rare de discord où une partie des joueurs /
bots disponibles disparaissent du selecteur @. Ce bug se règle tout
seul mais cela peut prendre du temps (heures voir jours). Cependant,
il est possible de contourner simplement le bug de deux façons. On
peut tout simplement utiliser un autre salon. Mais on peut aussi écrire
la commande dans un autre salon et la copier coller dans le salon
qui nous intéresse. Garder toujours un salon vide de toute écriture,
car c'est apparemment l'envoi de messages qui provoque cet étrange
bug.
\end{rem}

Une fois la commande exécutée, l'arbitre affiche le détail de la partie
proposée (voir la Figure~\ref{fig:waiting_start}). Les participants
doivent alors cliquer sur le ``pouce vers le haut'' en bas du message
pour valider la partie. Si vous utilisez un bot, cela est prévu par
la librairie, vous n'avez rien à faire. Dans la configuration actuelle,
les joueurs ont alors 15 minutes pour accepter la dite partie. Lorsque
tous les joueurs ont cliqué sur le pouce, la partie débute. Si vous
souhaitez décliner la proposition de partie: cliquez sur le ``pouce
vers le bas''.
\begin{rem}
Il est normal d'avoir le chiffre 1 pour les pouces haut et bas. C'est
l'arbitre qui a ajouté ces réactions pour les rendre visible et faciliter
leurs ajouts.
\end{rem}

\begin{figure}
\begin{centering}
\includegraphics[scale=0.5]{start_attente_de_reponse}
\par\end{centering}
\caption{Example of the beginning of a match (example of waiting for player
confirmation to start the game)}\label{fig:waiting_start}
\end{figure}


\subsubsection{Déroulement et fin de la partie}

Lors d'une partie, l'arbitre indique lorsqu'un joueur doit jouer.
Il écrit dans la conversation une phrase du type ``@<PLAYER> must
play (he has <TIME> left)''. A ce moment là, le joueur concerné peut
écrire dans la conversation l'action choisie par son programme. Un
joueur ne doit pas écrire avant qu'on lui a dit explicitement de jouer.
La Figure~\ref{fig:D=0000E9but-d'une-partie} représente un début
de partie, correspondant à la section d'action suivante : ``J4-I5''
puis ``C5'' (le premier joueur joue deux fois d'affilé). Remarquez
que l'arbitre valide les actions en les annotant par un carré vert.

\begin{figure}
\begin{centering}
\includegraphics[scale=0.5]{start}
\par\end{centering}
\caption{Example of the beginning of a match}\label{fig:D=0000E9but-d'une-partie}
\end{figure}

Si l'action écrite est incorrecte, l'action est annotée par un carré
rouge et l'arbitre signale l'erreur et fourni la liste des actions
licites (voir la Figure~\ref{fig:Invalid-move}).
\begin{figure}
\centering{}\includegraphics[scale=0.5]{invalid_move}\caption{Example of invalid move}\label{fig:Invalid-move}
\end{figure}

Lorsque la partie est finie, l'arbitre le signale et indique le vainqueur
(voir la Figure~\ref{fig:Example-of-endgame}).
\begin{figure}
\centering{}\includegraphics[scale=0.5]{endgame}\caption{Example of endgame}\label{fig:Example-of-endgame}
\end{figure}

Une fois la partie finie, vous pouvez lancer un nouveau match avec
la commande !start.

\subsubsection{Fin de match non standard: timeout / abandon}

Nous nous intéressons maintenant aux fins de parties spéciales, déclanchées
par un dépassement de temps de réflexion (par défaut 30 minutes par
partie par joueur) ou par la décision d'un joueur.

Si un joueur dépasse son temps de réflexion, la partie se termine
et ce joueur perd la partie (voir la Figure~\ref{fig:Example-of-timeout}).
\begin{figure}
\centering{}\includegraphics[scale=0.5]{timeout}\caption{Example of timeout}\label{fig:Example-of-timeout}
\end{figure}

Si vous avez besoin d'arrêter une partie en cours, vous pouvez utiliser
la commande ``!stop''. Cela va déclencher un vote. Cette commande
ne fonctionne pas pour le moment si vous utilisez un bot Discord.

Si vous souhaitez abandonner une partie, lorsque c'est à vous de jouer,
vous pouvez jouer l'action resign (écrivez ``resign'') . Cela va
mettre fin à la partie et faire gagner votre adversaire (sans déclancher
de vote). Si vous utilisez un bot: utilisez à la place la commande
``!resign''. Lorsque cela sera à votre bot de jouer, il jouera alors
l'action ``resign'' qui déclanchera ainsi la fin de la partie.

\subsubsection{Syntaxe des actions (format textuel)}\label{subsec:Syntaxe-des-actions}

Avec le DCOI, il faut bien distinguer deux choses, la syntaxe des
actions et les actions licites. Pour que le DCOI puisse différencier
une conversation normal d'une action de jeu, le DCOI a besoin d'une
syntaxe. Cette syntaxe inclut les actions du type ``A1'', ``B2-C12'',
mais aussi ``D5-E3-G15'' combinés à des mots clefs spécifiques,
par exemple ``queen-A8''. Notez que ``resign'' est un exemple
de mot clé commun à tous les jeux. Pour connaitre la liste des mots
clefs du jeu courant, il faut utiliser la commande:
\begin{center}
!show\_move\_keywords.
\par\end{center}

Notez que la syntaxe générale des actions de jeu est définie par l'expression
régulière suivante: 
\begin{center}
({[}a-zA-Z{]}{[}0-9{]}\{1,2\} | mot-clé-1 | ... | mot-clé-n)(-({[}a-zA-Z{]}{[}0-9{]}\{1,2\}
| mot-clé-1 | ... | mot-clé-n)){*}.
\par\end{center}

Autrement dit, une action au format textuel est une séquence de blocs
de longueur supérieure ou égale à 1, séparés par des tirets qui sont
soit une coordonnée (e.g. A12, B5, C18) soit un mot-clé.

Lorsqu'un texte est écrit dans la conversation, le bot arbitre vérifie
si le texte respecte la syntaxe attendue. Si c'est le cas et si c'est
à ce joueur de jouer, il va tenté de jouer cette action. Si cette
action est valide, il la joue, sinon il signale que l'action n'est
pas licite, comme décrit précédemment. Si la syntaxe n'est pas vérifiée,
l'arbitre ignore tout simplement le texte.

\subsection{Logging des matchs}\label{subsec:Logging-des-matchs}

Les parties sont sauvegardées dans des fichiers .json (dans le dossier
discord\_interface/log/bot\_play\_log/), au format 
\begin{center}
``play\_<PLAYER\_ID>\_Computer-Olympiad\_<CHANNEL\_NAME>\_<DATE>\_<TIME>.json''.
\par\end{center}

Ce fichier contient un dictionnaire Python lisible par un humain possédant
de nombreux attributs, et en particuliers les suivants: game\_name,
moves (history of moves of the match), players (list of PLAYER\_ID),
winner (True if you or your bot has won).

\subsection{Avertissement sur les parties trop courtes}\label{subsec:Avertissement-sur-les}

Notez qu'il y a une perte de temps lors de la communication des actions
dû à la latence internet (qui dépend de la qualité de votre connexion)
mais également de la latence de Discord et aussi dans une moindre
mesure de celle du DCOI. Cela se compte en secondes, en moyenne autour
d'une seconde. DCOI n'est donc pas adapté pour jouer des matchs avec
un temps moyen de 1 seconde par action. En outre, et de toute façon,
l'envoi de plus d'un message par seconde par un bot ou un utilisateur
enfreint le contrat d'utilisation de Discord qui intreprète un tel
comportement comme du flood. Ce comportement peut mener à diverses
punitions (bans, shadows bans, ralentissement des fonctions discord,
...). Avec l'envoi d'actions de l'ordre de la seconde, nous n'avons
cependant jamais rencontré de problèmes en pratique lors de nos nombreux
tests. 

\section{Jeu automatisé par bot Discord}\label{sec:Jeu-automatis=0000E9-par}

Dans cette section, nous allons expliquer les différentes façons de
réaliser un match avec le DCOI en mode automatique.

Dans tous les cas, le jeu automatisé s'effectue grâce à un bot Discord
joueur. Cela requiert de créer un ``compte'' bot Discord sur le
site développeur de Discord (voir la section~\ref{subsec:Cr=0000E9ation-du-compte}).
Ensuite, la façon native de faire des matchs en mode automatique est
d'étendre une des classes Player Python du DCOI (Section~\ref{subsec:Programmation-d'un-Bot-python}).
L'autre façon, décrite dans la Section~\ref{subsec:Utilisation-du-Bot-GTP},
est d'utiliser un bot déjà intégralement implémenté qui communique
avec votre programme grâce aux entrées-sorties (Inter-process communication)
en utilisant le Go Text Protocol (GTP). Cette deuxième méthode offre
donc l'avantage de ne pas requérir de programmation utilisant la librairie
du DCOI et donc de connaitre le langage Python.

Enfin, dans la Section~\ref{subsec:Fonctionnalit=0000E9s-avanc=0000E9es},
nous présentons des fonctionnalités avancées liées aux bots.

Nous commençons par détailler l'installation des packages Python requis
par le DCOI.

\subsection{Installation des packages requis du DCOI}

Nous détaillons maintenant l'installation du DCOI. Notez que Python
3.10+ est requis pour installer et utiliser le DCOI.

Pour effectuer l'installation des packages du DCOI, vous avez deux
méthodes:
\begin{rem}
Si vous utilisez Python pour vos travaux, afin de ne risquer de casser
les dépendances, nous vous invitons à utiliser un environnement virtuel
python.
\end{rem}


\paragraph{Méthode rapide:}

Exécutez dans le terminal: 
\begin{itemize}
\item ``cd discord\_interface''
\item ``chmod +x install.sh''
\item ``bash install.sh''
\item (optionnel) Test de l'installation: exécutez dans le terminal: ``python3
test\_installation.py''.
\end{itemize}
\begin{rem}
Si vous déplacez le fichier discord\_interface, vous devrez relancer
``bash install.sh''.
\end{rem}


\paragraph{Méthode pas à pas}

Le DCOI est basé sur la librairie Python officielle de Discord. Pour
pouvoir utiliser la librairie du DCOI, vous avez donc besoin de l'installer.
Pour ce faire, exécutez dans votre terminal: ``python3 -m pip install
-U discord.py''. Vous avez aussi besoin de la librairie numpy pour
faire tourner les jeux: ``python3 -m pip install -U numpy''. Si
vous souhaitez utiliser GTP, vous avez besoin en plus de la librairie
``pexpect'' pour permettre les communications inter-processus, exécutez
: ``pip install pexpect''. Vous pouvez ensuite tester l'installation
en exécutant dans le terminal: ``python3 test\_installation.py''.
\begin{rem}
Si vous utilisez GTP, si besoin, il est possible de se passer de la
dépendance à numpy: me contacter pour que je modifie le code en conséquence.
\end{rem}

Enfin, vous devez modifier le python path pour ajouter le dossier
parent de ``discord\_interface'':

\begin{verbatim}
UPPER_DIR_DCOI='<path_to_discord_interface>'
echo 'export PYTHONPATH=$PYTHONPATH:'$UPPER_DIR_DCOI >> ~/.bashrc
source ~/.bashrc
\end{verbatim}
\begin{rem}
Si vous déplacez le fichier discord\_interface, vous devrez modifier
le PYTHON PATH avec le nouveau chemin.
\end{rem}


\subsection{Création du compte de votre bot Discord Joueur}\label{subsec:Cr=0000E9ation-du-compte}

Nous nous intéressons maintenant à la création du compte de votre
bot, qui permet au programme du bot de se connecter à l'application
Discord. Plus précisément, cette série d'étapes permet de récupérer
diverses informations. Certaines de ces informations doivent être
saisies dans le fichier de configuration ``parameters.conf'' à la
racine du dossier discord\_interface. La création du compte permet
également de récupérer un lien à transmettre à l'organisateur pour
qu'il ajoute votre bot sur le serveur Discord de la compétition.

\subsubsection{Créer le compte du bot et inviter le bot dans le serveur}\label{subsec:Cr=0000E9er-le-compte-de-bot}
\begin{enumerate}
\item Allez sur le site web \url{https://discord.com/developers/applications}.
Il faut vous connecter avec votre compte discord. Vous devez ensuite
cliquer sur « New Application » en haut à droite de l’écran (si besoin,
voir la Figure~\ref{fig:New-Application}) et donner un nom au bot.
Vous arrivez alors sur la page du bot.
\begin{enumerate}
\item changez de navigateur internet en cas de difficulté
\end{enumerate}
\item Il faut ensuite activer l'option \textquotedbl Message Content Intent\textquotedbl{}
dans l'onglet Bot (si besoin, voir la Figure~\ref{fig:Bot-window-and}).
\item Puis, il faut récupérer le TOKEN du bot, c'est en quelque sorte son
mot de passe (il doit donc être gardé secret). Notez le quelque part,
nous en aurons besoin pour spécifier le fichier de configuration dans
la section suivante. 
\begin{enumerate}
\item Allez sur l'onglet Bot (toujours sur la page du bot). Dans la Section
TOKEN, cliquez sur le bouton ``Reset Token'' (si besoin, voir la
Figure~\ref{fig:Reset-Token-button}).
\end{enumerate}
\item Allez ensuite sur l'onglet OAuth2, section ``OAuth2 URL Generator'',
sous-section ``SCOPES''. Cochez l'option ``bot''. 
\item Puis dans la sous-section suivante ``bot permissions'', cochez les
options suivantes : \textquotedbl View Channels\textquotedbl , \textquotedbl Send
Messages\textquotedbl , \textquotedbl Read Message History\textquotedbl ,
\textquotedbl Add Reactions\textquotedbl . 
\item Pour terminer, allez en bas de la page, dans la zone ``Generated
URL'' et copier l'url. Transmettez cet url à l'organisateur pour
qu'il ajoute votre bot sur le serveur.
\end{enumerate}
\begin{figure}
\centering{}\includegraphics[scale=0.15]{new_application}\caption{New application button (red box)}\label{fig:New-Application}
\end{figure}
\begin{figure}
\centering{}\includegraphics[scale=0.15]{\string"message content intent\string".png}\caption{Bot window and message content intent option to be activated (red
boxes)}\label{fig:Bot-window-and}
\end{figure}
\begin{figure}
\centering{}\includegraphics[scale=0.15]{token}\caption{Reset Token button (red box)}\label{fig:Reset-Token-button}
\end{figure}


\subsubsection{Récupération de votre identifiant Discord (optionnel mais recommandé)}\label{subsec:R=0000E9cup=0000E9ration-de-votre}

Ensuite, il est suggéré de récupérer votre identifiant Discord pour
le spécifier dans le fichier de configuration. C'est optionnel, mais
cela permet d'utiliser des commandes contrôlant votre bot pour accéder
à certaines fonctionnalités. Par exemple, c'est requis pour pouvoir
reprendre un match en cours après un crash (voir la Section~\ref{subsec:Fonctionnalit=0000E9s-avanc=0000E9es}).
Nous décrivons dans cette section comment récupérer votre identifiant
Discord.

La première chose à faire est d'activer le mode développeur de son
compte discord. Pour se faire:
\begin{enumerate}
\item Ouvrir l'application Discord
\item Allez dans les paramètres (cliquez sur le bouton ``engrenage'',
si besoin voir la Figure~\ref{fig:Discord-parameters-button}). 
\item Cliquez sur l'onglet ``Advanced'' . 
\item Activez l'option ``developer mode''.
\item Cliquez en bas à droite sur votre nom. 
\item En bas, dans la mini-fenêtre, cliquez sur ``Copier l'identifiant
de l'utilisateur'' (si besoin voir la Figure~\ref{fig:Open-owner-window}).
\item Retournez dans les paramètres et désactivez l'option ``developer
mode''.
\end{enumerate}
\begin{figure}
\centering{}\includegraphics[scale=0.5]{interface_paramètres}\caption{Discord parameters button (red box)}\label{fig:Discord-parameters-button}
\end{figure}

\begin{figure}
\centering{}\includegraphics[scale=0.25]{advanced}\caption{Discord advanced window and developper mode option (red boxes)}\label{fig:Discord-advanced-window}
\end{figure}

\begin{figure}
\centering{}\includegraphics[scale=0.15]{\string"Owner id\string".png}\caption{Open owner window and copy Owner ID (red boxes)}\label{fig:Open-owner-window}
\end{figure}


\subsubsection{Spécification du fichier de configuration}

Nous décrivons maintenant comment spécifier le fichier de configuration,
``parameters.conf'' situé à la racine du dossier ``discord\_interface'',
qui permet au programme de bot d'accéder à Discord.
\begin{enumerate}
\item Spécifier la valeur de ``PLAYER\_BOT\_DISCORD\_TOKEN'' par le TOKEN
du compte du bot (récupéré lors de la Section~\ref{subsec:Cr=0000E9er-le-compte-de-bot}).
\item Optionnel mais recommandé, spécifier la valeur de ``OWNER\_ID''
par votre identifiant Discord (récupéré lors de la Section~\ref{subsec:R=0000E9cup=0000E9ration-de-votre}).
\end{enumerate}
Si vous souhaitez vérifier que le fichier de configuration est correctement
spécifié, lancez votre bot et faites:

\begin{verbatim}
cd <discord_interface_path>/discord_interface 
python3 main_random_ai.py
\end{verbatim}

S'il y a dans le terminal l'erreur ``Bad OWNER\_ID'', il faut donc
corriger la variable OWNER\_ID dans ``parameters.conf''.

Pour un test avancé, lancez votre bot et écrivez dans un salon du
serveur discord de la compétition:
\begin{center}
``!conf\_test''.
\par\end{center}
\begin{itemize}
\item Si rien ne se passe, contactez-moi.
\item Si tout marche bien, ``All is OK!'' est inscrit dans le terminal.
\item Si une erreur se déclanche dans le terminal (discord.errors.LoginFailure:
Improper token has been passed): le PLAYER\_BOT\_DISCORD\_TOKEN a
été mal spécifié.
\end{itemize}

\subsection{Programmation d'un bot Discord grâce à Python}\label{subsec:Programmation-d'un-Bot-python}

Nous décrivons maintenant comment programmer un bot Discord en Python.
Il faut hériter une des classes Python ``Player'' proposées dans
la librairie du DCOI. Nous décrivons ici la manière la plus simple
de le faire.

\paragraph{Programmation du bot}

Pour programmer votre bot, il faut hériter la classe BasicPlayer du
fichier ``discord\_interface/player/model/basic\_player.py''. La
seule chose que vous avez à faire est de définir la méthode my\_plays(self,
game\_history, time\_left=inf, opponent\_time\_left=inf). Cette méthode
doit retourner l'action que veut jouer votre programme dans l'état
courant. L'état courant peut-être reconstruit à partir du paramètre
game\_history, qui contient la liste des actions au format textuel
joué depuis le début de la partie. Il est également possible d'utiliser
les méthodes de jeux natives fournies par le DCOI, ce qui permet de
ne pas avoir besoin d'implémenter le jeu. Dans ce second cas d'usage,
vous aurez besoin des méthodes décrites dans la Table~\ref{tab:Elementary-game-methods}.

\begin{table}
\begin{centering}
\begin{tabular}{|c|c|}
\hline 
game methods of any player & \tabularnewline
\hline 
\hline 
self.game.get\_numpy\_board() & return a numpy array representing the game board\tabularnewline
\hline 
self.game.get\_current\_player() & returns $i$ if the current player is the player $i$ ($i\in\mathbb{N}$) \tabularnewline
\hline 
self.game.textual\_legal\_moves() & list of valid actions in text format (e.g. ``A1'' or ``A2-B3'')\tabularnewline
\hline 
self.game.ended() & return True si the game is ended\tabularnewline
\hline 
self.game.winner & return the index $i$ of the winner player ($i\in\mathbb{N}$) \tabularnewline
\hline 
self.game.textual\_plays(move) & apply the move in the current game state (i.e.self.game)\tabularnewline
\hline 
self.game.undo() & undo the last action of the current game state (i.e.self.game)\tabularnewline
\hline 
\end{tabular}
\par\end{centering}
\caption{Elementary methods of any Player Python object for programming an
AI player.}\label{tab:Elementary-game-methods}

\end{table}

Faites attention à la valeur du paramètre ``time\_left'', il indique
le temps qu'il reste à votre programme pour le reste du match. Si
cette valeur atteint 0, vous perdez la partie. 

Un exemple d'héritage est disponible dans la classe TextualRandomAI
du fichier 
\begin{center}
``discord\_interface/player/instances/textual\_random\_ai.py''. 
\par\end{center}

Ce dernier est un bot qui joue aléatoirement sauf s'il y a un coup
gagnant parmi les actions disponibles. Il le joue alors.

\paragraph{Lancement du bot}

Une fois le compte de votre bot créé, le fichier de configuration
spécifié en conséquence, et la classe Player hérité, il ne reste qu'à
lancer votre bot dans un fichier main. Pour cela, il faut utiliser
la méthode bot\_starting(AI\_Class) du fichier discord\_interface/player/model/bot\_launcher.py
avec pour AI\_Class votre AI Player. Un exemple de programme lanceur
de bot est disponible dans le fichier 
\begin{center}
``discord\_interface/main\_random\_ai.py''.
\par\end{center}

Il ne vous reste plus qu'à faire ``python3 votre\_main.py'' dans
le dossier ``discord\_interface''. Par exemple:

\begin{verbatim}
cd <discord_interface_path>/discord_interface 
python3 main_textual_random_ai.py
\end{verbatim}

Le bot attend désormais que la commande !s @<Bot\_name\_1> @<Bot\_name\_2>
soit lancée et que le bot arbitre affiche les informations de la partie.
A ce moment là, le bot effectue son initialisation puis ajoute son
pouce. La partie commence et le bot joue automatiquement lorsque c'est
à lui, jusqu'à la fin de la partie.
\begin{rem}
Si vous avez utilisé un environnement virtuel python lors de l'installation,
n'oubliez pas de l'activer pour lancer le bot.
\end{rem}


\paragraph{Fonctionnalités avancées}

Pour finir, optionnellement, si vous avez besoin de fonctionnalités
avancées, comme un code à effectuer à l'initialisation ou à la terminaison
du match, vous pouvez hériter de la classe AdvancedBasicPlayer (fichier
discord\_interface/player/model/advanced\_basic\_player.py) et ainsi
définir les méthodes my\_initialisation(game\_name) et my\_end().

\subsection{Utilisation du bot Discord Joueur générique grâce au GTP}\label{subsec:Utilisation-du-Bot-GTP}

Nous décrivons maintenant comment réaliser des matchs de manière automatique
en utilisant GTP. Des informations à propos de GTP sont disponibles
par exemple sur ce site: \url{https://www.lysator.liu.se/~gunnar/gtp/gtp2-spec-draft2/gtp2-spec.html}.
Nous verrons dans la Section~\ref{subsec:Sp=0000E9cification-des-commandes}
comment spécifier un programme GTP. Nous commençons par décrire comment
lancer le bot GTP, supposant que nous avons déjà un programme déjà
spécifié au format GTP. Pour finir, nous verrons une fonction avancée
permettant d'afficher des informations de votre programme dans le
terminal (Section~\ref{subsec:Afficher-des-informations}) mais également
comment gérer les erreurs de votre programme avec GTP (Section~\ref{subsec:Gestion-des-erreurs}).

\subsubsection{Lancement du bot GTP}

Nous commençons par expliquer les pré-requis pour lancer votre bot
GTP. Premièrement, il faut connecter votre programme au bot GTP. Vous
avez donc besoin de spécifier trois paramètres:
\begin{itemize}
\item ``program\_name'' : le nom de votre programme joueur,
\item ``program\_arguments'': la liste des arguments de votre programme
(séparés par des espaces),
\item ``program\_directory'': le chemin absolu de votre programme (ou
le chemin relatif au dossier discord\_interface, par exemple vous
pouvez utiliser une chaine vide si le programme se trouve à la racine
du dossier ``discord\_interface'').
\end{itemize}
Ces variables doivent être spécifiées dans le fichier de configuration
``parameters.conf'' à la racine du dossier ``discord\_interface''.

Pour exécuter votre programme, vous n'avez plus qu'à faire 
\begin{center}
``python3 main\_gtp\_ai\_from\_conf.py''
\par\end{center}

dans le terminal depuis la racine du dossier ``discord\_interface''.
\begin{rem}
Si vous êtes à l'aise avec Python, vous pouvez aussi modifier et exécuter
le fichier ``main\_gtp\_ai.py'' pour éviter de toucher au fichier
de configuration.
\end{rem}


\subsubsection{Spécification des commandes GTP}\label{subsec:Sp=0000E9cification-des-commandes}

Afin de pouvoir utiliser le bot basé GTP, il faut que votre programme
lise dans l'entrée standard les commandes GTP (qui seront envoyées
par le bot GTP) et écrive dans la sortie standard le résultat de ces
commandes (qui seront ainsi lues et traitées par le bot GTP). Les
commandes GTP requises pour faire tourner le bot GTP, à implémenter
dans votre programme, sont décrites dans la Table~\ref{tab:List-of-GTP-commands}.
Pour vérifier que la spécification des commandes GTP est correcte,
vous pouvez lancer le programme de test suivant dans le terminal:
``python3 test\_gtp\_ai\_from\_conf.py''.

\begin{table}
\begin{centering}
\begin{tabular}{|c|c|c|}
\hline 
GTP command & return & effect\tabularnewline
\hline 
\hline 
clear\_board & $=$ & restart the game\tabularnewline
\hline 
undo & $=$ & cancel the last move\tabularnewline
\hline 
play <player> <move> & $=$ & plays <move> for <player>\tabularnewline
\hline 
time\_left <player> <time> & $=$ & inform the time left in seconds\tabularnewline
\hline 
genmove <player> & $=$<move> & return the <move> to play and play it\tabularnewline
\hline 
quit & $=$ & shutdown the program\tabularnewline
\hline 
\hline 
player & $=$<player> & return the current player\tabularnewline
\hline 
game <game> & $=$ & initialization for <game>\tabularnewline
\hline 
\end{tabular}
\par\end{centering}
\caption{List of GTP commands to implement, the syntax of their return and
their expected effect (<move> is an action in text format (see the
syntax in Section~\ref{subsec:Syntaxe-des-actions}) ; <player> is
an integer $n\in\mathbb{N}$ ; <game> is the game name of the match).
Remark: commands ``game'' and ``player'' are not part of standard
GTP ; they are optional (make them only return ``='') ; ``game''
is required only for multi-games participation ; ``player'' is required
only for profiling (see Section~\ref{subsec:Bot-profiling})..}\label{tab:List-of-GTP-commands}

\end{table}

Si besoin, un exemple de spécification en Python des commandes GTP
est disponible dans le fichier ``discord\_interface/player/instances/autres/gtp\_random\_ai.py''.
\begin{rem}
Exemple de communication GTP entre le bot et le programme joueur:

\begin{verbatim}
genmove 0
=A2-A3
play 1 A7-A6
=
genmove 0
=A3-A4
...
\end{verbatim}
\end{rem}

\begin{rem}
Si vous souhaitez que votre bot ait la capacité d'abandonner automatiquement,
il suffit qu'il retourne ``=resign'' suite à la commande ``genmove'',
au moment approprié. 
\end{rem}

\begin{rem}
Si vous souhaitez faire abandonner votre bot manuellement, utilisez
la commande déjà fonctionnelle ``!resign''. 
\end{rem}


\subsubsection{Afficher des informations dans le terminal}\label{subsec:Afficher-des-informations}

Pour terminer, nous parlons brièvement d'une option qui permet, lors
de l'utilisation du bot GTP, d'afficher dans le terminal des informations
provenant de son programme.

La mise en place de cette fonctionnalité est très simple. A chaque
return de commande GTP, il suffit de concaténer au return le symbole
``\#'' suivi de la ligne à afficher dans le terminal. 
\begin{example}
Si l'on veut afficher dans le terminal ``Le coup a été correctement
annulé'' lorsque la commande ``undo'' est exécutée, le return du
undo doit être ``=\#Le coup a été correctement annulé''.
\end{example}

Notez que la ligne à afficher dans le terminal ne doit pas contenir
de saut de ligne. Il est toutefois possible d'effectuer des sauts
de lignes avec l'ajout de plusieurs ``\#''. Chaque ``\#'' supplémentaire
sera alors interprété comme un saut de ligne.
\begin{example}
Si le return de la commande genmove est ``=<move>\#Calcul du prochain
coup terminé\#Coup joué <move>\#Temps du calcul 15 secondes'', cela
affichera dans le terminal:
\end{example}

\begin{verbatim}
Calcul du prochain coup terminé
Coup joué <move>
Temps du calcul 15 secondes
\end{verbatim}

\subsubsection{Gestion des erreurs avec GTP}\label{subsec:Gestion-des-erreurs}

Si vous souhaitez avoir une gestion des erreurs avec GTP, c'est-à-dire
que votre programme transmette au bot Discord qu'il y a eut une erreur,
il faut utiliser la syntaxe GTP classique prévue à cette effet. Plus
précisément, lorsqu'une erreur se produit et doit être communiquée,
il faut écrire dans la sortie standard mais en modifiant le format
de sortie en remplaçant ``=<output\_line>'' par ``?<output\_line>''.
La présence du ``?'' permettra au bot Discord de lancer une erreur,
qui sera notamment affichée dans le terminal. D'autres informations
seront également affichées dans le terminal :
\begin{itemize}
\item le nom de la commande ayant déclanchée le ``?'', 
\item la sortie de votre programme et 
\item l'affichage des commentaires (si vous avez mis en place la fonctionnalité
de la section précédente).
\end{itemize}

\subsection{Fonctionnalités avancées}\label{subsec:Fonctionnalit=0000E9s-avanc=0000E9es}

Nous décrivons maintenant plusieurs fonctionnalités avancées des bots
joueurs.

\subsubsection{Gestion des crashs }

Si pour une raison quelconque votre bot vient à planter et doit être
re-exécuté. Il faut pour reprendre le match en cours, après avoir
relancé son bot, utiliser la commande 
\begin{center}
``!continue''.
\par\end{center}

Son alias est ``!c''.

Avant la compétition, je vous suggère d'effectuer un kill en cours
de partie, de relancer et de vérifier que la commande !continue fonctionne
correctement pour être sûr de pouvoir reprendre une partie en cas
de problème.

Si le bot arbitre ne réagit pas, contactez un responsable de la compétition
pour qu'il relance l'arbitre et exécute également la commande ``!continue''.

\subsubsection{Bot profiling}\label{subsec:Bot-profiling}

Du profilage peut être effectué avec la commande ``!p''. 
\begin{rem}
Si vous utilisez un bot GTP, certaines des valeurs fournies seront
incorrectes si la commande GTP ``player'' n'a pas ou a été mal programmée.
\end{rem}


\section{Création d'un serveur de jeu Discord personnel}\label{sec:Cr=0000E9ation-d'un-serveur}

Nous expliquons dans cette section comment créer son serveur de jeu
personnel, pour faire une beta test local pour les Computer Olympiad
ou pour organiser sa propre compétition. Plusieurs étapes doivent
être réalisées : création d'un serveur Discord (Section~\ref{subsec:Cr=0000E9ation-d'un-serveur}),
création du ``compte de bot'' pour le bot arbitre (Section~\ref{subsec:Cr=0000E9ation-du-compte-arbitre}),
invitation des joueurs sur le serveur (Section~\ref{subsec:Invitation-des-joueurs}),
invitation des bots sur le serveur (Section~\ref{subsec:Invitation-des-bots}),
et enfin, lancement du bot arbitre (Section~\ref{subsec:Lancement-du-bot}).

\subsection{Création d'un serveur Discord}\label{subsec:Cr=0000E9ation-d'un-serveur}

Nous commençons donc par détailler la création d'un serveur Discord:
\begin{enumerate}
\item Dans la barre verticale de vos serveurs Discord, située à gauche de
l'application Discord:
\begin{enumerate}
\item cliquez sur le ``+'', puis ``créer le mien'' (choisissez vos options
et son nom).
\end{enumerate}
\item Activez le ``mode développeur'' (étapes 1 à 4 de la Section~\ref{subsec:R=0000E9cup=0000E9ration-de-votre}).
\item Clic gauche sur l'icone du serveur créé dans la barre des serveurs.
\item Cliquez sur ``Copier l'identifiant du serveur''.
\item Pour une beta test des Computer Olympiad:
\begin{enumerate}
\item Dans le fichier ``parameters.conf'':
\begin{enumerate}
\item dupliquez la ligne commençant par:
\begin{enumerate}
\item ``BETA\_TEST\_COMPUTER\_OLYMPIAD\_GUILD\_ID=''.
\end{enumerate}
\item commentez la ligne originale avec ``\#''. 
\item utilisez l'identifiant du serveur créé pour modifier la valeur de:
\begin{enumerate}
\item ``BETA\_TEST\_COMPUTER\_OLYMPIAD\_GUILD\_ID''.
\end{enumerate}
\item verifiez que dans le fichier ``parameters.conf'', nous avons bien: 
\begin{enumerate}
\item ``BETA\_TEST\_MODE=True''.
\end{enumerate}
\end{enumerate}
\item pour chaque bot arbitre (d'index $i$):
\begin{enumerate}
\item dans le serveur, créer un salon ``match\_\textsl{i}''
\item clic droit sur le salon: copier ID:
\item Dans le fichier ``parameters.conf'':
\begin{enumerate}
\item collez comme valeur de REFEREE\_BOT\_\textsl{i}\_DISCORD\_CHANNEL
\end{enumerate}
\end{enumerate}
\end{enumerate}
\item Pour un serveur à autre usage:
\begin{enumerate}
\item Dans le fichier ``parameters.conf'':
\begin{enumerate}
\item dupliquez la ligne commençant par:
\begin{enumerate}
\item ``COMPUTER\_OLYMPIAD\_GUILD\_ID=''.
\end{enumerate}
\item commentez la ligne originale avec ``\#''. 
\item utilisez l'identifiant du serveur récupéré pour modifier la valeur
de:
\begin{enumerate}
\item ``COMPUTER\_OLYMPIAD\_GUILD\_ID''.
\end{enumerate}
\item verifiez que dans le fichier ``parameters.conf'', nous avons bien:
\begin{enumerate}
\item ``BETA\_TEST\_MODE=False''.
\end{enumerate}
\end{enumerate}
\item pour chaque bot arbitre (d'index $i$):
\begin{enumerate}
\item dans le serveur, créer un salon ``match\_\textsl{i}''
\item clic droit sur le salon: copier ID:
\item Dans le fichier ``parameters.conf'':
\begin{enumerate}
\item collez comme valeur de BETA\_TEST\_REFEREE\_BOT\_\textsl{i}\_DISCORD\_CHANNEL
\end{enumerate}
\end{enumerate}
\end{enumerate}
\item Désactivez le ``mode développeur''.
\end{enumerate}
\begin{rem}
Si vous participez ensuite à la beta test des Computer Olympiad, n'oubliez
pas de supprimer la ligne ajoutée et décommenter l'originale.
\end{rem}

\begin{rem}
Si vous participez ensuite aux Computer Olympiad, n'oubliez pas de
mettre ``BETA\_TEST\_MODE=False''.
\end{rem}


\subsection{Création du ``compte de bot'' arbitre}\label{subsec:Cr=0000E9ation-du-compte-arbitre}

Nous expliquons maintenant comment créer un ``compte de bot'' pour
un bot arbitre. 

La procédure à suivre est identique à celle de la Section~\ref{subsec:Cr=0000E9er-le-compte-de-bot}
sauf pour l'étape 5 où il faut en plus activer l'option ``Manage
Messages''.

\subsection{Invitation des joueurs sur le serveur}\label{subsec:Invitation-des-joueurs}

Nous expliquons maintenant comment inviter les joueurs et opérateurs
de bot humain sur votre serveur:
\begin{enumerate}
\item Clic droit sur l'icone de votre serveur dans la barre verticale des
serveurs à gauche.
\item Cliquez sur ``Inviter des gens''.
\item Cliquez sur le bouton ``Copier''.
\item Envoyez aux personnes à inviter, le lien ainsi copié.
\end{enumerate}
Les personnes à inviter devront alors suivre le protocole de la Section~\ref{subsec:Discord:-serveurs-et-salons}
en utilisant ce lien à la place du lien des Computer Olympiad.

\subsection{Invitation des bots sur le serveur}\label{subsec:Invitation-des-bots}

Nous expliquons maintenant comment inviter les bots joueurs et arbitre
sur votre serveur. Nous supposons que vous avez suivi la Section~\ref{subsec:Cr=0000E9ation-d'un-serveur}
et ainsi récupéré l'URL d'invitation du bot arbitre et que les participants
vous ont également fourni leur URL d'invitation après avoir suivi
la Section~\ref{subsec:Cr=0000E9ation-du-compte}. Pour chaque URL:
\begin{enumerate}
\item Coller l'URL dans votre navigateur. 
\item Connectez vous à votre compte discord (si requis). 
\item Choisissez votre serveur parmi la liste des serveurs proposés.
\item Cliquez sur le bouton ``Continuer''. 
\item Si c'est un bot d'une autre personne:
\begin{enumerate}
\item vérifiez que les autorisations demandées correspondent bien à celles
spécifiées dans la section correspondante de création du compte de
bot (joueur: Section~\ref{subsec:Cr=0000E9ation-du-compte} / arbitre:
Section~\ref{subsec:Cr=0000E9ation-d'un-serveur}).
\end{enumerate}
\item Cliquez sur le bouton ``Autoriser''.
\end{enumerate}

\subsection{Lancement du bot arbitre}\label{subsec:Lancement-du-bot}

Nous expliquons, dans cette section, comment lancer le bot arbitre.
Pour ce faire, exécutez: 
\begin{center}
``python3 main\_referee.py''
\par\end{center}

dans le terminal depuis la racine du dossier ``discord\_interface''.

Les joueurs en mode manuel n'ont plus qu'à suivre la Section~\ref{sec:Jeu-manuel}
et les joueurs en mode automatique la Section~\ref{sec:Jeu-automatis=0000E9-par}.
\begin{rem}
Si le programme du bot arbitre venait à planter en plein match, relancer
le, puis exécutez la commande ``!continue'' dans le salon du match.
\end{rem}

\begin{rem}
Si l'un des bots joueurs venait à planter en plein match et que le
relancer et utiliser ``!continue'' ne règle pas le problème, vous
pouvez tenter de killer l'arbitre, le relancer et utiliser la commande
``!continue'' (si vous supposez que le problème provenait ou a été
propagé au bot arbitre).
\end{rem}


\end{document}
